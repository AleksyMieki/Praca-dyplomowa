\chapter{Testy}

\section{Sprawdzanie poprawności funkcji}

Aplikacja z racji muzycznej jej natury, w dużej mierze wymaga wizualnej oceny odnośnie sposobu działania. Większość testów przeprowadzona została manualnie. Działanie aplikacji było na bieżąco kontrolowane poprzez funkcję \emph{Play}, uruchamiając aktualny stan sceny wraz z podłączonymi komponentami.

Aby zapewnić poprawność działania sporządzono plan testów obejmujący każdy zawarty w~aplikacji widok. Plan prezentował się następująco:
\begin{itemize}
    \item przetestowanie działania każdego z przycisków sceny głównej,
    \item przetestowanie działania metronomu, przy 5 różnych ustawieniach parametrów, w tym: różne wartości tempa i metrum,
    \item przetestowanie działania przycisków koła kwintowego, upewniając się, że dla każdego z nich wyświetlana jest adekwatna skala,
    \item przetestowanie stroika, strojąc za jego pomocą gitarę, następnie porównując poziom nastrojenia za pomocą zewnętrznego stroika, aplikacji \emph{GuitarTuna},
    \item przetestowanie działania treningu słuchu korzystając z analizatora dźwięku, aby upewnić się, że odegrane dźwięki odpowiadają przyciskom.
\end{itemize} 

\paragraph{Testy metronomu}

Testy obejmowały sprawdzenie poprawności działania przycisków, oraz funkcji odpowiedzialnej za wątek główny. Posłużono się zewnętrznym narzędziem \texttt{Metronome} \cite{Musicca}, aby porównać działanie aplikacji. Wątek oceniono pod względem tego, jak długo trwa on w~zależności od ustalonego tempa gry. Korzystając z funkcji \texttt{Debug.Log()} silnika Unity, sprawdzono czas trwania każdego poszczególnego "obiegu" wątku. Przy ilości taktów wynoszących 3,5,7,8 i bpm odpowiednio 60,60,80,100 narzędzie to nie ujawniło żadnych usterek w działaniu.

\paragraph{Testy księgi akordów}

Test księgi akordów polegał na wizualnym porównaniu aktualnie wyświetlanego akordu, z bazą wszystkich możliwych akordów narzędzia mobilnej wersji aplikacji \emph{GuitarTuna}.  Przeanalizowano wszystkie możliwe kombinacje, aby upewnić się, że grafiki wyświetlają się poprawnie, a akordy umieszczone zostały w prawidłowych pozycjach.

\paragraph{Stroika}

Testy stroika przeprowadzone zostały ręcznie, z użyciem gitary, na której odgrywane były kolejne nuty w naturalnym strojeniu(E1,B,G,D,A,E2). Wyniki analizowane były pod kątem dokładności, korzystając przy tym z narzędzia stroika \emph{GuitarTuna}, którego to wyniki porównywane były do rezultatów uzyskiwanych w aplikacji. Same testy wykazały poprawność działania, umożliwiając poprawne nastrojenie gitary. 

\paragraph{Testy treningu słuchu}

Testy narzędzia treningu słuchu przeprowadzone zostały za pomocą zewnętrznego narzędzia, służącego do analizy dźwięku. Dzięki niemu, podczas odgrywania nuty przez aplikację narzędzie analizowało odegrany dźwięk, na podstawie którego wybierano odpowiadający mu przycisk z nutą. Takich prób przeprowadzono 24, dla każdej nuty dwukrotnie, aby zagwarantowac poprawność działania funkcji oceniającej odpowiedź.

\paragraph{Testy koła kwintowego}

Aby przetestować działanie koła kwintowego sprawdzono poprawność działania każdego z 12 przycisków. Oceniono przy tym, czy wyświetlana po jego naciśnięciu skala muzyczna zgadza się z przyjętą teorią muzyki. Podczas testów, każdy z przycisków wykazał poprawność działania, wyświetlając przy tym 12 nut odpowiadającej danej skali i 3 odpowiadające akordom pasującym do wybranego dźwięku. 