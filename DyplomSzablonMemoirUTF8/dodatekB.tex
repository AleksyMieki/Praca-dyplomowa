\chapter{Instrukcja obsługi aplikacji}

Niniejszy rozdział zawiera instrukcję obsługi dla aplikacji \emph{Guitarrr}. 

Po uruchomieniu aplikacji użytkownikowi zaprezentowane jest menu główne z 6 przyciskami:

\begin{itemize}
    \item Metronom -- po naciśnięciu użytkownik zostanie przeniesiony do widoku metronomu.
    \item Akordy gitarowe -- po naciśnięciu użytkownik zostanie przeniesiony do widoku księgi akordów.
    \item Stroik -- po naciśnięciu użytkownik zostanie przeniesiony do widoku stroika.
    \item Trening słuchu -- po naciśnięciu użytkownik zostanie przeniesiony do widoku treningu słuchu.
    \item Koło kwintowe -- po naciśnięciu użytkownik zostanie przeniesiony do widoku koła kwintowego.
    \item X -- po naciśnięciu aplikacja zostanie wyłączona.
\end{itemize}

\subsection{Metronom}

Korzystając z metronomu użytkownik ma dostęp do 2 przycisków odpowiedzialnych za zmianę ilości taktów (początkowo ustawione na 2), oraz 2 przyciski odpowiedzialne za zmianę ilości bpm(początkowo ustawione na 60). Aby dodać takt należy nacisnąć przycisk z napisem "Dodaj takt", natomiast aby odjąć tak użytkownik winien nacisnąć przycisk "Odejmij takt". Przycisk ,,+'' służy do zwiększenia liczby bpm, natomiasst ,,-'' do ich zmniejszenia. Jeżeli użytkownik chciałby rozpocząć trening należy nacisnąć przycisk ,,Start''.

\subsection{Księga akordów}

