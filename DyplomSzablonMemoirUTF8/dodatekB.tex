\chapter{Instrukcja obsługi aplikacji}
Po uruchomieniu aplikacji \emph{Guitarrr} w jej oknie pojawi się menu główne z 6 przyciskami:
\begin{itemize}
    \item Metronom -- po naciśnięciu użytkownik zostanie przeniesiony do widoku metronomu.
    \item Akordy gitarowe -- po naciśnięciu użytkownik zostanie przeniesiony do widoku księgi akordów.
    \item Stroik -- po naciśnięciu użytkownik zostanie przeniesiony do widoku stroika.
    \item Trening słuchu -- po naciśnięciu użytkownik zostanie przeniesiony do widoku treningu słuchu.
    \item Koło kwintowe -- po naciśnięciu użytkownik zostanie przeniesiony do widoku koła kwintowego.
    \item X -- po naciśnięciu aplikacja zostanie wyłączona.
\end{itemize}

Każdy z widoków zawiera przycisk ze znakiem "?", po naciśnięciu którego wyświetlona zostanie pomoc kontekstowa.

\paragraph{Metronom}

Korzystając z metronomu, użytkownik ma dostęp do 2 przycisków odpowiedzialnych za zmianę liczby taktów (początkowo ustawione na 2) oraz 2 przyciski odpowiedzialne za zmianę liczby BPM (początkowo ustawione na 60). Aby dodać takt, należy nacisnąć przycisk z napisem "Dodaj takt", natomiast aby odjąć takt, użytkownik winien nacisnąć przycisk "Odejmij takt". Przycisk "+" służy do zwiększenia liczby BPM, natomiast "-" do ich zmniejszenia. Jeżeli użytkownik chciałby rozpocząć trening, należy nacisnąć przycisk "Start".

\paragraph{Księga akordów}

Korzystając z księgi akordów na ekranie widoczny jest akord, który podpisany jest nad nim. Do zmiany akordu służą 4 przyciski, które użytkownik może naciskać, aby przeglądać listę dostępnych akordów. Poniżej akordu znajduje się przycisk "Zapisz". Po jego naciśnięciu aktualnie wyświetlany akord zostanie zapisany i będzie dostępny w widoku zapisanych akordów.

\paragraph{Diagram akordów}

Diagram akordów prezentuje wszystkie dostępne w aplikacji akordy. Użytkownik może przewijać diagram za pomocą dwóch przycisków umieszczonych na dole ekranu. Po naciśnięciu na jeden z akordów użytkownik zostanie przeniesiony do widoku księgi akordów.

\paragraph{Stroik}

W widoku stroika dla użytkownika dostępne jest 6 przycisków oznaczających nuty gitarowe. Po naciśnięciu jednego z przycisków rozpocznie się faza nagrywania. Wtedy użytkownik powinien wziąć swoją gitarę i uderzyć w docelową strunę. Pośrodku sceny podświetlony zostanie 1 z 3 elementów (patrząc od lewej), oznaczających: zbyt niski ton, dobry ton lub zbyt wysoki ton.

\paragraph{Trening słuchu}

Scena treningu słuchu wita użytkownika wyborem 4 możliwych tonacji do odbycia treningu. Po wybraniu jednej z 4 możliwych tonacji przed użytkownikiem pojawi się odpowiadająca skala majorowa. Następnie należy nacisnąć przycisk "GRAJ", po czym odgadnąć odegraną nutę za pomocą dostępnych przycisków. W razie chęci zmiany skali należy nacisnąć przycisk reset. Po jego naciśnięciu ponownie wyświetlone zostaną 4 możliwe tonacje.

\paragraph{Koło kwintowe}

Koło kwintowe zawiera 12 przycisków zewnętrznych, z którymi użytkownik może wchodzić w~interakcję. Po naciśnięciu jednego z przycisków na zewnętrznym kole, kolorowo zaznaczone zostaną te nuty, które wchodzą w skład skali majorowej, dla której wybrana nuta jest tonem głównym. Ponadto na szaro podświetlą się przyciski symbolizujące akordy pasujące do danej skali.
