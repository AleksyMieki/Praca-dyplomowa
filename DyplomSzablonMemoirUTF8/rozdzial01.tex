\chapter{Wstęp}
\section{Wprowadzenie}
Muzyka towarzyszyła ludziom od zarania dziejów. Pierwotne instrumenty przybierały postać prostych bębnów czy grzechotek, używanych podczas rytuałów dawnych plemion. Wraz z rozwojem cywilizacji muzyka również ewoluowała, przybierając różne formy – od bluesa granego przez uciemiężonych niewolników, po muzykę klasyczną, popularną w bogatszych kręgach Europy. Instrumenty stawały się coraz bardziej skomplikowane, a sama muzyka coraz bardziej powszechna i dostępna dla zwykłych ludzi.

Dzięki rozwojowi technologii wytwarzanie instrumentów przestało być drogim przedsięwzięciem. Bogata rynkowa oferta, jak również przystępna cena sprawia, że większość osób może sobie na ich zakup pozwolić. Łatwiejszy zaś dostęp do instrumentów sprzyja rozwojowi dziedziny, w tym dodatkowych narzędzi oraz usług.
Dotyczy to między innymi edukacji muzycznej.

W przypadku nauki gry na gitarze edukację tę zazwyczaj kojarzy się z tradycyjnymi lekcjami udzielanymi przez nauczycieli. Podczas serii spotkań przekazują oni swoją wiedzą uczniom, zaczynając od podstaw, monitorując postępy. Inną popularną metodą nauki stały się książki, które często zawierają jedynie diagramy akordów czy skal. Niestety, wiedza w nich zawarta bywa trudna do przyswojenia i szybko się dezaktualizuje, z uwagi na ciągły rozwój muzyki, zmiany styli oraz powstawanie nowych utworów.

Naturalnym krokiem w rozwoju nauki gry, wraz z rozwojem Internetu, stały się kursy online, aplikacje mobilne oraz portale dostępne on-line, które oferują ogrom informacji na temat teorii muzyki, skal muzycznych tworzonych przez wieki oraz transkrypcji popularnych piosenek. Niestety, wraz ze wzrostem liczby dostępnych informacji, wzrosła też ilość materiałów niskiej jakości, powtarzających powszechnie znane treści lub oferujących mało wartościowe, fragmentaryczne lekcje. Problem ten częściowo rozwiązują aplikacje do nauki gry, które zawierają jedynie najważniejsze funkcje, przydatne dla graczy.

Do najpopularniejszych aplikacji tego typu należą Yousician, Fret Trainer czy Guitar Tuna. Oferują one łatwy dostęp do kluczowych narzędzi, które wspomagają naukę gry, takich jak metronom, stroik oraz wbudowane gry umożliwiające naukę akordów i skal. Aplikacje te mają jednak istotną wadę – oferują one bezpłatnie jedynie ograniczony wgląd do pełnych wersji piosenek przechowywanych w bazach danych, dostęp do większość z nich jest odpłatny.

Istnieją również strony internetowe oferujące rozpiski wszystkich możliwych skal gitarowych, jednak nie oferują one metod nauki poza samym wyświetleniem diagramów skal na gryfie. Nauka gry na gitarze nie jest też intuicyjna dla początkujących. Istnieje niewiele materiałów, które w przystępny sposób prezentują schematy ułożenia nut na gryfie czy zasady budowania skal. W edukacyjnych materiałach często pomijane są naturalne korelacje między dźwiękami wynikające z ich częstotliwości. Z tego powodu wielu początkujących graczy rezygnuje po kilku próbach nauki, gdyż mimo zapamiętania wielu akordów i piosenek nie są w stanie zrobić kroku dalej w kierunku zrozumienia istoty gry i tworzenia własnych utworów z braku odpowiednich narzędzi.

Problem nie znika nawet po opanowaniu podstaw i teorii potrzebnej do komponowania własnych utworów. Pojawia się wtedy trudność w tworzeniu progresji akordów. Mimo zaznajomienia z typowymi akordami w skalach, trudno zapamiętać wszystkie dostępne permutacje. Jedynym sposobem na sprostanie temu wyzwaniu jest wyszukiwanie akordów i notowanie ich "na boku", ponieważ nie istnieje narzędzie służące do tworzenia progresji, a istnieje wiele sposobów na aranżację nawet prostych przejść akordowych.

Biorąc pod uwagę powyższe aspekty i brak zunifikowanego systemu nauki, postanowiłem stworzyć aplikację desktopową, łączącą w sobie najważniejsze aspekty teorii i praktyki, aby umożliwić graczom łatwy sposób przyswajania teorii muzyki, oferując jednocześnie wszystkie istotne narzędzia potrzebne do nauki i samej gry na gitarze.

\section{Cel pracy}
Celem tej oto pracy dyplomowej jest stworzenie aplikacji desktopowej zawierającej wszystkie potrzebne narzędzia służące przyswajaniu teorii oraz umożliwiającej łatwą praktykę samej gry. Aplikacja ma na celu ułatwienie użytkownikom procesu nauki w sposób prosty, zwracając uwagę na najbardziej istotne aspekty gry oraz wypłaszczyć próg wejścia do gry dla początkujących graczy, a dla graczy bardziej zaawansowanych ułatwić proces kompozycji własnych utworów. 

Aby osiągnąć wyżej postawiony cel aplikacja powinna zawierać:
\begin{itemize}
    \item grę polegającą na wyświetlaniu na ekranie nut na gryfie, które następnie użytkownik ma rozpoznać
    \item metronom
    \item stroik monochromatyczny
    \item wizualną reprezentację koła kwintowego
    \item listę dostępnych akordów wraz z ich permutacjami oraz możliwość zapisu do 5 wybranych akordów, które następnie można będzie wyświetlić w oddzielnym oknie
\end{itemize}

\section{Zakres pracy}
Zakres pracy obejmuje:
\begin{itemize}
    \item Analiza wymagań
    \begin{itemize}
        \item Zidentyfikowanie problemów, z jakimi mierzą się początkujący jak i średniozaawansowani gitarzyści, takie jak trudności w nauce teorii, akordów, skal oraz progresji. 
        \item Określenie wymagań funkcjonalnych i niefunkcjonalnych. Ustalenie, jakie funkcje powinny się znaleźć w aplikacji
    \end{itemize}
    \item Przegląd istniejących rozwiązań
    \begin{itemize}
        \item Analiza dostęnych aplikacji pod kątem tego jakie funkcjonalności zawierają, ustalenie ich zalet oraz wad, przeprowadzenie na tej podstawie analizy odnośnie tego co zawierać ma docelowa aplikacja
    \end{itemize}
    \item Projektowanie aplikacji
    \begin{itemize}
        \item Stworzenie makiety aplikacji w oprogramowaniu FIGMA
        \item Stworzenie graficznego interfejsu użytkownika
        \item Opracowanie architektury aplikacji
    \end{itemize}
    \item Implementacja aplikacji
    \begin{itemize}
        \item Ustalenie technologii w jakiej aplikacja będzie realizowana
        \item Zaimplementowanie ustalonych wcześniej funkcjonalności wraz z interfejsem użytkownika
    \end{itemize}
    \item Testowanie aplikacji
    \begin{itemize}
        \item Testy funkcjonalne - sprawdzenie poprawności działania poszczególnych funkcjonalności aplikacji
        \item Testy manualne - przeprowadzenie testów sprawdzających intuicyjność oraz użyteczność zaimplementowanych funkcjonalności
    \end{itemize}
    \item Dokumentacja aplikacji
    \begin{itemize}
        \item Sporządzenie instrukcji użytkownika, opisującej dogłębnie sposób korzystania z aplikacji
        \item Sporządzenie dokumentacji technicznej aplikacji
    \end{itemize}
\end{itemize}


\section{Układ dokumentu}
W rozdziale pierwszym przedstawiono w zarysie czym jest i czego dotyczy niniejszy dokument (jest to szablon, który można zastosować podczas redagowania pracy dyplomowej inżynierskiej bądź magisterskiej). W rozdziale drugim opisano sposób pracy z szablonem. W kolejnym, trzecim rozdziale, przedstawiono zalecenia dotyczące formatowania dokumentu. Rozdział ten pełni rolę czysto informacyjną (dostarczony szablon zapewnia uzyskanie opisanego tam formatowania).
W rozdziale czwartym zwrócono uwagę na redakcję pracy dyplomowej (od strony edytorskiej i merytorycznej).
Rozdział piąty poświęcono na uwagi techniczne. Ostatni, szósty rozdział, przeznaczono na kilka słów podsumowania oraz ,,lorem ipsum'' -- wygenerowany tekst, pełniący rolę ,,wypełniacza'', wykorzystany w celach poglądowych (jak dzielić dokument na sekcje).
Pracy towarzyszy przykładowy wykaz literatury oraz przykładowe dwa dodatki. 

