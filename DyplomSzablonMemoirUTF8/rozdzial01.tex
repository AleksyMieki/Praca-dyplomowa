\chapter{Wstęp}

\section{Wprowadzenie}
Już od zarania dziejów muzyka towarzyszyła ludziom w dobrych i złych chwilach. Moc niesionego przez nią przekazu często zwiększano poprzez wykorzystanie różnego rodzaju instrumentów. Pierwsze z nich przyjmowały postać prostych bębnów i grzechotek, pełniących ważną rolę podczas rytuałów plemiennych, albo piszczałek czy też okaryn, zabieranych na wędrówki. Wraz z rozwojem cywilizacji instrumenty te zaczęto doskonalić, sama zaś muzyka ewoluowała, przybierając różne formy -- od bluesa, granego przez uciemiężonych niewolników, po muzykę klasyczną, popularną w bogatszych kręgach Europy. 

%Tworzenie i słuchanie muzyki z biegiem lat stało się czynnością powszechną, dostępna dla zwykłych ludzi.

Dzięki zastosowaniu nowoczesnych technologii wytwarzanie instrumentów nie jest już drogim przedsięwzięciem. Skutkiem tego na rynku pojawiła się bogata ich oferta. Mnogość zaś oferowanych modeli, jak również przystępna cena  instrumentów sprawiały, że wiele osób może sobie na ich zakup pozwolić. Łatwiejszy zaś dostęp do instrumentów sprzyja rozwojowi dziedziny, w tym dodatkowych narzędzi oraz usług. Dotyczy to między innymi edukacji muzycznej.

W przypadku nauki gry na gitarze edukację tę zazwyczaj kojarzy się z tradycyjnymi lekcjami udzielanymi przez nauczycieli. Podczas serii spotkań przekazują oni swoją wiedzą uczniom, zaczynając od podstaw, monitorując postępy. Inną popularną metodą nauki stały się książki, które często zawierają jedynie diagramy akordów czy skal. Niestety, wiedza w nich zawarta bywa trudna do przyswojenia i szybko się dezaktualizuje, z uwagi na ciągły rozwój muzyki, zmiany styli oraz powstawanie nowych utworów.

Naturalnym krokiem w rozwoju nauki gry, wraz z rozwojem Internetu, stały się kursy online, aplikacje mobilne oraz portale dostępne on-line, które oferują ogrom informacji na temat teorii muzyki, skal muzycznych tworzonych przez wieki oraz transkrypcji popularnych piosenek. Niestety, wraz ze wzrostem liczby dostępnych informacji, wzrosła też ilość materiałów niskiej jakości, powtarzających powszechnie znane treści lub oferujących mało wartościowe, fragmentaryczne lekcje. Problem ten częściowo rozwiązują aplikacje do nauki gry, które zawierają jedynie najważniejsze funkcje, przydatne dla graczy.

Do najpopularniejszych aplikacji tego typu należą \texttt{Yousician} \verb+\cite{https://yousician.com}+ czy \texttt{Fret Trainer} \verb+\cite{https://fretmaster.app/}+. % TO DO: dołożyć cytowania (do opisów tych aplikacji, manuali czy książek
Oferują one łatwy dostęp do kluczowych narzędzi, które wspomagają naukę gry, takich jak metronom, stroik oraz wbudowane gry umożliwiające naukę akordów i skal. Aplikacje te mają jednak istotną wadę – oferują one bezpłatnie jedynie ograniczony wgląd do pełnych wersji piosenek przechowywanych w bazach danych, dostęp do większość z nich jest odpłatny.

Istnieją również strony internetowe oferujące rozpiski wszystkich możliwych skal gitarowych, jednak nie oferują one metod nauki poza samym wyświetleniem diagramów skal na gryfie. Nauka gry na gitarze nie jest też intuicyjna dla początkujących. Istnieje niewiele materiałów, które w przystępny sposób prezentują schematy ułożenia nut na gryfie czy zasady budowania skal. W edukacyjnych materiałach zwykle pomijane są naturalne korelacje między dźwiękami wynikające z ich częstotliwości. Stąd też wielu początkujących graczy rezygnuje po kilku próbach podjęcia nauki, gdyż mimo zapamiętania wielu akordów i piosenek nie są w stanie zrobić kroku dalej w kierunku zrozumienia istoty gry i tworzenia własnych utworów. Przyczynia się też do tego brak odpowiednich narzędzi.

Problem nie znika nawet po opanowaniu podstaw i teorii potrzebnej do rozpoczęcia pierwszych kompozycji. Pojawia się wtedy trudność w tworzeniu progresji akordów. Mimo zaznajomienia z typowymi akordami w skalach, trudno zapamiętać wszystkie dostępne permutacje. Jedynym sposobem na sprostanie temu wyzwaniu jest wyszukiwanie akordów i notowanie ich ,,na boku'' (istnieje wiele sposobów na aranżację nawet prostych przejść akordowych, jednak przy braku wspierających narzędzia do tworzenia progresji ręczne zapiski są jedyną opcją).

Biorąc pod uwagę powyższe aspekty i brak zunifikowanego systemu nauki w ramach niniejszej pracy postanowiono zająć się stworzeniem aplikacji desktopowej, łączącej w sobie najważniejsze aspekty teorii i praktyki. Dzięki niej osoby rozpoczynające swoją przygodę z grą na instrumentach powinny w łatwiejszy sposób przyswajać sobie teorii muzyki, mając do dyspozycji wszystkie istotne narzędzia potrzebne do nauki i samej gry na gitarze.

\section{Cel i zakres pracy}
Celem pracy jest stworzenie aplikacji desktopowej dostarczającej zbiór narzędzi wspierających przyswajanie teorii muzyki oraz umożliwiającej praktykowanie gry na instrumencie. Aplikacja ma ułatwiać muzyczną edukację początkującym graczom poprzez kierowanie ich uwagi na najbardziej istotne aspekty gry, łagodząc przy tym próg wejścia w poszczególne tematy, zaś graczom bardziej zaawansowanym -- ma pomóc w procesie kompozycji własnych utworów. 

Myśląc o jak najlepszym sposobie osiągnięcia wyznaczonego celu w trakcie realizacji pracy na sam początek postanowiono przyjrzeć się rozwiązaniom istniejącym już na rynku od dłuższego czasu, takim jak wymienionych wcześniej już \texttt{Guitar Tuna} czy \texttt{Tunefire}. % DONE - jak to się ma do trzech aplikacji wymienionych we wprowadzeniu? 
Analiza tych aplikacji pozwolić na zidentyfikowanie ich najlepszych cech, największych wad, a następnie wykorzystanie zdobytej wiedzy podczas realizacji autorskiego projektu. 


Jako kolejny krok zaplanowano określenie wymagań funkcjonalnych i niefunkcjonalnych, rozpoznanie narzędzi najbardziej odpowiadającym gitarzystom, ustaleniem wyglądu i designu interfejsu użytkownika. Na bazie tych założeń nastąpić ma implementacja aplikacji.

Do projektowania makiet interfejsu użytkownika posłużyć ma oprogramowanie Figma. Z~jego pomocą zostanie przygotowana wstępna wizja widoków aplikacji. Ponadto oprogramowanie to zostanie użyte do stworzenia grafik potrzebnych dla przycisków, do reprezentacji akordów, oraz pozostałych elementów takich jak napisy, grafiki, tytuły poszczególnych scen. 

Za makietami musi działać jakaś logika biznesowa zgodnie z przyjętym modelem architektury. W ramach projektu trzeba będzie określić zasady komunikacji pomiędzy elementami scen, układ scen, systemy ładowania elementów podczas przemieszczania pomiędzy widokami w aplikacji. Mając gotowy projekt, makietę oraz architekturę nastąpi faza implementacji aplikacji za pomocą silnika Unity, z wykorzystaniem języka C\#, korzystając z IDE Rider.

Po zaimplementowaniu wszystkich funkcjonalności aplikacja poddana zostanie testom. Z racji muzycznej natury aplikacji wymagane będzie tutaj skorzystanie z zewnętrznego instrumentu. Większość testów przeprowadzane będzie manualnie. Ponadto części kodu aplikacji zostanie pokryta testami jednostkowymi, przetestowane będą też poszczególne komponenty. 

W ramach pracy sporządzona zostanie pełna dokumentacja projektu, razem z instrukcją dla nowo wdrażających się użytkowników, aby uniknąć niejasności podczas pierwszego spotkania z aplikacją. 


\section{Układ dokumentu}
Rozdział pierwszy poświęcony został na wstęp, zaznajomienie czytelnika z tematem pracy, wyjaśnieniem potrzebnych konceptów, wytłumaczenie celu pracy oraz określenie zakresu jaki owa praca obejmuje. 

Rozdział drugi zawiera analizę istniejących rozwiązań, niezbędną do stworzenia własnej aplikacji. Analiza przedstawia wyciągnięte z istniejących rozwiązań najlepsze cechy, wraz z wadami, które następnie zostały wzięte pod uwagę podczas etapu projektowania.

W rozdziale 3 przedstawiony został projekt aplikacji, zawiera on: wymagania funkcjonalne, niefunkcjonalne, projekt graficznego interfejsu użytkownika, oraz projekt architektury produktu.

