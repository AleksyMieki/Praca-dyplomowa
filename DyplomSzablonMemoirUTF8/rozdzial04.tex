\chapter{Implementacja}

\section{Wprowadzenie do implementacji}

Rozdział ten zawiera szczegóły implementacji aplikacji. Każdej scen poświęcono odpowiadający podrozdział wyjaśniający wykorzystane technologie, biblioteki oraz sposób komunikacji pomiędzy poszczególnymi elementami widoku. 

\section{Implementacja elementów aplikacji}


\subsubsection{Stroik}
                                                                                                                                                              
\subsubsection{Metronom}

\subsubsection{Księga akordów}

Scena księgi akordów zawiera element graficzny reprezentujący aktualnie wybrany akord gitarowy, przyciski umożliwiające wybór docelowego akordu oraz elementy graficzne ułatwiające nawigację po widoku sceny.

Głównym skryptem realizującym logikę sceny jest ChordChartsManager przypięty do menadżera sceny. Zawiera on 14 pól obiektów :

\begin{itemize}
    \item noteText
    \item variantText
    \item noteList
    \item variantList
    \item noteLeft
    \item noteRight
    \item variantLeft
    \item variantRight
    \item chordImage
    \item image
    \item Chords
    \item ch
    \item noteIterator
    \item variantIterator
\end{itemize}

Funkcje wchodzące w skład skryptu to:

\begin{itemize}
    \item Start - Funkcja odpowiedzialna za przygotowanie potrzebnych elementów sceny. Przypisuje grafiki akordów z pomocniczej listy Chords do listy głównej ch. Ustawia wartości zmiennych noteIterator, variantIterator z wartości przechowywanych w skrypcie SceneConnector, łączącej scenę księgi akordów z widokiem wszystkich akordów w postaci diagramu. Przypisuje ona funkcje do odpowiednich przycisków i resetuje wartości obiektów tekstowych.
    \item noteIncrement - Funkcja przypisana do przycisku, odpowiedzialna za zmianę nuty akordu, inkrementując wartość aktualnie wybranej nuty.
    \item noteDecrement - Funkcja przypisana do przycisku, odpowiedzialna za zmianę nuty akordu, dekrementując wartość aktualnie wybranej nuty.
    \item variantIncrement - Funkcja przypisana do przycisku, odpowiedzialna za zmianę wariacji akordu, inkrementując wartość aktualnie wybranej wariacji.
    \item variantDecrement - Funkcja przypisana do przycisku, odpowiedzialna za zmianę wariacji akordu, dekrementując wartość aktualnie wybranej wariacji.
    \item ChangeVariantText - Funkcja odpowiedzialna za zmianę przypisanej wartości obiektu tekstowego sceny, wykorzystywana w momencie zmiany wariacji akordu.
    \item ChangeNoteText - Funkcja odpowiedzialna za zmianę przypisanej wartości obiektu tekstowego sceny, wykorzystywana w momencie zmiany nuty akordu.
    \item ChangeChord - Funkcja odpowiedzialna za zmianę aktualnie wyświetlanego akordu, zmienia przypisaną grafikę obiektu Image sceny.

\end{itemize}

\subsubsection{Trening słuchu}

Scena treningu słuchu zawiera przycisk odpowiedzialny za odegranie dźwięku, który następnie ma zostać odgadnięty przez użytkownika za pomocą jednego z 12 przycisków obecnych na scenie. 
W skład sceny wchodzą dwa skrypty, EarTrainerManager - przypisany do menadżera sceny realizujący główną logikę sceny oraz ButtonIndex - skrypt pomocniczy przypisany do każdego z 12 przycisków odpowiedzialnych za udzielenie odpowiedzi.

Główny skrypt zawiera 6 pól zmiennych:

\begin{itemize}
    \item noteClips 
    \item playNoteButton  
    \item answerButtons  
    \item feedbackText  
    \item audioSource 
    \item correctNoteIndex
\end{itemize}

Funkcje wchodzące w skład głównego skryptu to:

\begin{itemize}
    \item Start - Funkcja przypisująca obiekt audioSource, odpowiedzialny za odgrywanie dźwięków. Ustawia ona również funkcję PlayRandomNote do przycisku playNoteButton. Dodatkowo przypisuje ona funkcję NoteGuessed do każdego z przycisków odpowiedzialnych za odpowiedź.
    \item PlayRandomNote - Funkcja odpowiedzialna za odegranie losowego dźwięku po naciśnięciu przycisku playNoteButton. Ustawia ona odpowiedni index dla odegranej nuty, wykorzystywany następnie podczas zweryfikowania odpowiedzi podanej przez użytkownika.
    \item NoteGuessed - Funkcja weryfikująca odpowiedź użytkownika, wykorzystywana po naciśnięciu jednego z 12 przycisków odpowiedzi. Ustawia ona obiekty tekstowe sceny, w zależności od tego czy odpowiedź była poprawna, czy też nie. 
\end{itemize}

Skrypt pomocniczy przechowuje jedynie jedną zmienną, wykorzystywaną w momencie weryfikacji odpowiedzi udzielonej przez użytkownika, jest nią: buttonNumber.

\subsubsection{Koło kwintowe}

Scena koła kwintowego zawiera 24 przyciski, każdemu z nich przypisana została odpowiednia nuta. W skrypcie menadżera sceny utworzone zostały dwie listy, przechowujące przyciski koła zewnętrznego oraz koła wewnętrznego. Skrypt zawiera w sobie 6 funkcji:

\begin{itemize}
\item Start - Funkcja używana podczas tworzenia obiektu, służy ona przygotowaniu wszystkich przycisków oraz przypisania im działania przy ich naciśnięciu, w tym przypadku przypisuje ich działanie do funkcji DisplayScale
\item DisplayScale - Funkcja odpowiedzialna jest ona za wyświetlenie skali majorowej, oraz 3 akordów minorowych pasujących do wybranej przez użytkownika nuty. 
\item GetColorTint - Funkcja pomocnicza generująca kolejne odcienie kolorów, dla funkcji DisplayScale. 
\item GetMajorScaleIndices - Funkcja pomocnicza generująca interwały kolejnych nut w skali majorowej
\item ChangeButtonColor - Funkcja pomocnicza odpowiedzialna za zmianę koloru odpowiedniego przycisku
\item ClearColors - Funkcja pomocnicza czyszcząca kolory wszystkich przycisków
\end{itemize}

