\chapter{Implementacja}

\section{Wprowadzenie do implementacji}

%TODO wrzucić listingi

Rozdział ten zawiera szczegóły implementacji aplikacji. Każdej scen poświęcono odpowiadający podrozdział wyjaśniający wykorzystane technologie, biblioteki oraz sposób komunikacji pomiędzy poszczególnymi elementami widoku. 

\section{Implementacja elementów aplikacji}

\subsubsection{Stroik}

Głównymi elementami sceny sa przyciski służące do wyboru domniemanej nuty. W skład widoku wchodzi tez jej tytuł, informator odnośnie aktualnego poziomu nastrojenia instrumentu i przyciski odpowiedzialne za zmianę docelowego instrumentu. Do menadżera sceny przypięte zostały 3 skrypty realizujące logikę potrzebną do nagrania, przeanalizowania i wyświetlenia wyników analizy dźwięku. Nazwy tych skryptów to: \emph{Audio Recorder} - skrypt odbierający dźwięk ze sceny, \emph{Audio Pitch Estimator} - odpowiedzialny za analizę odebranego dźwięku, \emph{Tuner Manager} - główny skrypt realizujący logikę sceny. Skrypt \emph{Audio Recorder} zawiera dwa pola:

\begin{itemize}
    \item audioSource - obiekt przechowujące odwołanie do wbudowanego w silnik Unity narzędzia \emph{AudioSource}, umożliwiającego nagranie dźwięku,
    \item duration - zmienna odpowiedzialna za ustalenie czasu, przez jaki mikrofon ma odbierać dźwięk z otoczenia.
\end{itemize}

Skrypt zawiera jedną funkcję \emph{OnEnable}, która podczas uruchomienia zaczyna odbierać dźwięk dostarczany do mikrofonu. Skrypt \emph{AudioPitchEstimator}, zaczerpnięty z repozytorium \cite{https://github.com/nakakq/AudioPitchEstimatorForUnity}, służy do analizy odebranego dźwięku. Biblioteka zawiera poniżej wymienione zmienne, na które użytkownik może wpłynąć, aby dopasować narzędzie do swoich potrzeb.

\begin{itemize}
    \item frequencyMin - minimalna częstotliwość, jaka ma być analizowana,
    \item frequencyMax - maksymalna częstotliwość poddawana analizie,
    \item harmonicsToUse - ilość harmonicznych jakie mają być brane pod uwagę, podczas ustalania końcowej częstotliwości odebranego dźwięku,
    \item smoothingWidth - pasmo częstotliwości filtra wygładzającego widmo,
    \item thresholdSRH - wartość graniczna ilości decybeli, od której dźwięk ma być wzięty pod analizę.
\end{itemize}

Funkcją realizującą główną logikę działania jest metoda \emph{Estimate}, wykonująca szybką transformatę Fouriera na odebranym dźwięku. Funkcja ta zwraca zmienną float \emph{bestFreq}, będąca najlepszym możliwym według programu "trafieniem", jeżeli chodzi o odebraną częstotliwość. Skrypt \emph{TunerManager} odpowiedzialny za realizowanie logiki sceny, przechowuje odwołania do komponentów sceny: przyciski, pola tekstowe, obiekt \emph{AudioSource}. W jego skład wchodzi 6 funkcji o nazwie \emph{SetDesiredFrequency*}, przy czym każdemu z przycisków przypisany jest jeden jej odpowiednik, ustawiający parametry skryptu \emph{AudioPitchEstimator}. Funkcją odpowiedzialną za logikę analizy dźwięku jest \emph{EstimatePitch}, która korzysta ze skryptu \emph{AudioPitchEstimator}, na etj podstawia ustawia wartości elementów tekstowych informujących użytkownika o poziomie nastrojenia instrumentu. Poza tym istnieją tu jeszcze dwie metody:

\begin{itemize}
    \item Start - Funkcja przypisująca przyciskom im działania w wypadku naciśnięcia.
    \item ClearColor - Funkcja pomocnicza, odpowiedzialna za "czyszczenie" elementów graficznych sceny.
\end{itemize}

\subsubsection{Metronom}
Scena metronomu zawiera 9 przycisków służących do parametryzacji działania metronomu, umożliwiając zwiększenie tempa, zmianę metrum, zwiększanie lub zmniejszanie liczby taktów. Za logikę odpowiada menadżer sceny z przypiętym skryptem \emph{MetronomeManager}, zawierający 10 pól obiektów przycisków, pole obiektu tekstowego i kilka zmiennych pomocniczych. Metronom realizowany jest za pomocą wizualizacji poszczególnych taktów za pomocą graficznych elementów w postaci "kółek", które dodawane sa do widoku dynamicznie w zależności od ustalonych przez użytkownika wartości. Funkcje jakie implementuje dany skrypt to:

\begin{itemize}
    \item Start - Funkcja odpowiadająca za ustawienie zmiennych, obiektów widoku. 
    \item AddCircle - Funkcja wykorzystywana do zwiększenia tempa.
    \item DeleteCircle - Funkcja wykorzystywana do zmniejszenia tempa.
    \item RearrangeElements - Funkcja stworzona do rozmieszczania "kółek" taktów na ekranie.
    \item StartMetronome - Funkcja uaktywniająca metronom.
    \item RecolorAll - Funkcja uruchamiana podczas resetu metronomu, czyszcząca grafiki wizualizacji metronomu.
    \item MetronomeCoroutine - Osobny wątek odpowiedzialny za działanie metronomu, w czasie działania tej funkcji wybijane jest na ekranie tempo.
    \item AddBpm - Funkcja odpowiedzialna za zwiększenie tempa.
    \item SubtractBpm - Funkcja odpowiedzialna za zmniejszenie tempa.
    \item ChangeMeasure* - Funkcje, przypisane do odpowiednich przycisków, w zależności od wyboru całej nuty, pół nuty, ćwierć nuty czy ósemki, odpowiednio ustawiają ilość bić na takt.
\end{itemize}
\subsubsection{Księga akordów}

Scena księgi akordów zawiera element graficzny reprezentujący aktualnie wybrany akord gitarowy, przyciski umożliwiające wybór docelowego akordu, elementy graficzne ułatwiające nawigację po widoku sceny. Funkcje wchodzące w skład skryptu menadżera sceny nazwanego \emph{ChordChartsManager} to:

\begin{itemize}
    \item Start - Funkcja odpowiedzialna za przygotowanie potrzebnych elementów sceny. Przypisuje grafiki akordów z pomocniczej listy Chords do listy głównej ch. Ustawia wartości zmiennych noteIterator, variantIterator z wartości przechowywanych w skrypcie SceneConnector, łączącej scenę księgi akordów z widokiem wszystkich akordów w postaci diagramu. Przypisuje ona funkcje do odpowiednich przycisków i resetuje wartości obiektów tekstowych.
    \item noteIncrement - Funkcja przypisana do przycisku, odpowiedzialna za zmianę nuty akordu.
    \item noteDecrement - Funkcja przypisana do przycisku, odpowiedzialna za zmianę nuty akordu.
    \item variantIncrement - Funkcja przypisana do przycisku, odpowiedzialna za zmianę wariacji akordu.
    \item variantDecrement - Funkcja przypisana do przycisku, odpowiedzialna za zmianę wariacji akordu.
    \item ChangeVariantText - Funkcja odpowiedzialna za zmianę przypisanej wartości obiektu tekstowego sceny, wykorzystywana w momencie zmiany wariacji akordu.
    \item ChangeNoteText - Funkcja odpowiedzialna za zmianę przypisanej wartości obiektu tekstowego sceny, wykorzystywana w momencie zmiany nuty akordu.
    \item ChangeChord - Funkcja odpowiedzialna za zmianę aktualnie wyświetlanego akordu, zmienia przypisaną grafikę obiektu Image sceny.
\end{itemize}

\subsubsection{Trening słuchu}

Scena treningu słuchu zawiera przycisk odpowiedzialny za odegranie dźwięku, który następnie ma zostać odgadnięty przez użytkownika za pomocą jednego z 12 przycisków obecnych na scenie. W skład sceny wchodzą dwa skrypty, \emph{EarTrainerManager} - przypisany do menadżera sceny realizujący główną logikę sceny oraz \emph{ButtonIndex} - skrypt pomocniczy przypisany do każdego z 12 przycisków odpowiedzialnych za udzielenie odpowiedzi. Funkcje wchodzące w skład głównego skryptu to:

\begin{itemize}
    \item Start - Funkcja przypisująca obiekt audioSource, odpowiedzialny za odgrywanie dźwięków. Ustawia ona również funkcję \emph{PlayRandomNote} do przycisku \emph{playNoteButton}. Dodatkowo przypisuje ona funkcję \emph{NoteGuessed} do każdego z przycisków odpowiedzialnych za odpowiedź.
    \item PlayRandomNote - Funkcja odpowiedzialna za odegranie losowego dźwięku po naciśnięciu przycisku \emph{playNoteButton}. Ustawia ona odpowiedni index dla odegranej nuty, wykorzystywany następnie podczas zweryfikowania odpowiedzi podanej przez użytkownika.
    \item NoteGuessed - Funkcja weryfikująca odpowiedź użytkownika, wykorzystywana po naciśnięciu jednego z 12 przycisków odpowiedzi. Ustawia ona obiekty tekstowe sceny, w zależności od tego czy odpowiedź była poprawna, czy też nie. 
\end{itemize}

Skrypt pomocniczy przechowuje jedynie jedną zmienną, wykorzystywaną w momencie weryfikacji odpowiedzi udzielonej przez użytkownika, jest nią: buttonNumber.

\subsubsection{Koło kwintowe}

Scena koła kwintowego zawiera 24 przyciski, każdemu z nich przypisana została odpowiednia nuta. W skrypcie menadżera sceny utworzone zostały dwie listy, przechowujące przyciski koła zewnętrznego oraz koła wewnętrznego. Skrypt zawiera w sobie 6 funkcji:

\begin{itemize}
\item Start - Funkcja używana podczas tworzenia obiektu, służy ona przygotowaniu wszystkich przycisków oraz przypisania im działania przy ich naciśnięciu, w tym przypadku przypisuje ich działanie do funkcji \emph{DisplayScale}
\item DisplayScale - Funkcja odpowiedzialna jest ona za wyświetlenie skali majorowej, oraz 3 akordów minorowych pasujących do wybranej przez użytkownika nuty. 
\item GetColorTint - Funkcja pomocnicza generująca kolejne odcienie kolorów, dla funkcji \emph{DisplayScale}. 
\item GetMajorScaleIndices - Funkcja pomocnicza generująca interwały kolejnych nut w skali majorowej
\item ChangeButtonColor - Funkcja pomocnicza odpowiedzialna za zmianę koloru odpowiedniego przycisku
\item ClearColors - Funkcja pomocnicza czyszcząca kolory wszystkich przycisków
\end{itemize}

