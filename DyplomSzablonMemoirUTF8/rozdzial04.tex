\chapter{Implementacja}

W niniejszym rozdziale przedstawiono szczegóły implementacji aplikacji. Każdej ze scen poświęcono osobny podrozdział, w którym odniesiono się do wykorzystanych technologii, bibliotek oraz zrealizowanych sposobów komunikacji pomiędzy poszczególnymi elementami widoku. W projekcie zamieszczono kody funkcji realizujących najistotniejsze zadania.

\section{Struktura aplikacji}
Struktura projektu wykorzystuje domyślne ułożenie plików w systemie Unity. Jako główny folder służy folder \texttt{Assets}, zawiera on główne pliki projektowe: skrypty, tekstury, dźwięki, grafiki, prefaby(gotowe elementy często wykorzystywane w projekcie).

Pliki w folderze \texttt{Scripts} zorganizowano według ich celu. Każdy plik przypisany do danej sceny znajduje się w odpowiednim folderze (rys.~\ref{fig:strukturaKodu}). Elementy pomocnicze, jak np.\ kod pomocy kontekstowej, kod przycisku zamykania sceny, znajdują się w folderze \texttt{Utility}. 
 
\begin{figure}[htb]
     \centering
 	\includegraphics[scale=.8]{rys04/StrukturaPlikow}
	\caption{Struktura źródeł kodu w projekcie}
	\label{fig:strukturaKodu}
\end{figure}

% DONE: w rozdziale dobrze byłoby zamieścić opis struktury projektu. Ponieważ rozwijał Pan aplikację w środowisku UNITY, to środowisko narzuca trochę tę strukturę (parę uwag na ten temat też by się przydało). Strukturę projektu można pokazać robiąc zrzuty drzewa źródeł kodu.

% DONE: przydałoby się też opowiedzieć trochę o budowaniu aplikacji (testowe uruchomienia odbywają się w środowisku, aby można było uruchomić aplikację poza tym środowiskiem - coś trzeba przecież zrobić).
Aby na bieżąco analizować poprawność zaimplementowanych funkcji wykorzystywano wbudowaną w silnik Unity opcję \emph{play}, która umożliwia ,,odpalenie'' jednej sceny, nie wymuszając przy tym budowania całej aplikacji od nowa. Po wykryciu zmian w kodzie aplikacja się przeładowuje i jest gotowa do użycia/przetestowania przez użytkownika.

\section{Implementacja elementów aplikacji}

\subsection{Stroik}

Głównymi elementami sceny sa przyciski służące do wyboru domniemanej nuty. W skład widoku wchodzi tez jej tytuł, informator odnośnie aktualnego poziomu nastrojenia instrumentu i~przyciski odpowiedzialne za zmianę docelowego instrumentu. Do menadżera sceny przypięte zostały 3 pliki z kodem zawierające definicje klas, realizujące logikę potrzebną do nagrania, przeanalizowania i wyświetlenia wyników analizy dźwięku. Nazwy tych klas to: \texttt{AudioRecorder} - klasa odbierająca dźwięk ze sceny, \texttt{AudioPitchEstimator} - odpowiedzialna za analizę odebranego dźwięku, \texttt{TunerManager} - główna klasa realizująca logikę sceny. Widok zawiera też przyciski odpowiedzialne za uruchomienie pomocy kontekstowej i opuszczenie bieżącej sceny.

Klasa \texttt{AudioRecorder} zawiera dwa pola:
\begin{itemize}
    \item \texttt{audioSource} -- obiekt przechowujące odwołanie do wbudowanego w silnik Unity narzędzia \texttt{AudioSource}, umożliwiającego nagranie dźwięku,
    \item \texttt{duration} -- zmienna odpowiedzialna za ustalenie czasu, przez jaki mikrofon ma odbierać dźwięk z otoczenia.
\end{itemize}
Klasa ta zawiera również jedną funkcję \texttt{OnEnable}, która podczas uruchomienia zaczyna odbierać dźwięk dostarczany do mikrofonu. 

Klasa \texttt{AudioPitchEstimator}, zaczerpnięta z repozytorium \cite{AudioPitchEstimator}, % DONE: to nie jest poprawne cytowanie, proszę zrobić odpowiedni rekord w bib, a potem użyć jego klucza w \cite
służy do analizy odebranego dźwięku. Zawiera ona wymienione poniżej zmienne, na które użytkownik może wpłynąć, aby dopasować narzędzie do swoich potrzeb.
\begin{itemize}
    \item \texttt{frequencyMin} -- minimalna częstotliwość, jaka ma być analizowana,
    \item \texttt{frequencyMax} -- maksymalna częstotliwość poddawana analizie,
    \item \texttt{harmonicsToUse} -- ilość harmonicznych jakie mają być brane pod uwagę, podczas ustalania końcowej częstotliwości odebranego dźwięku,
    \item \texttt{smoothingWidth} -- pasmo częstotliwości filtra wygładzającego widmo,
    \item \texttt{thresholdSRH} -- wartość graniczna decybeli, od której dźwięk ma być wzięty pod analizę.
\end{itemize}

Funkcją uruchamiającą szybką transformatę Fouriera na odebranym dźwięku jest \texttt{Estimate} (listing~\ref{lst:1}) z tej klasy. Zwraca one zmienną float \texttt{bestFreq}, będąca najlepszym możliwym według programu ,,trafieniem'', jeżeli chodzi o odebraną częstotliwość. Klasa \texttt{TunerManager} odpowiedzialna za realizowanie logiki sceny, przechowuje odwołania do komponentów sceny: przyciski, pola tekstowe, obiekt \texttt{AudioSource}. W jej skład wchodzi 6 funkcji o nazwie \texttt{SetDesiredFrequency*}, przy czym każdemu z przycisków przypisany jest jeden jej odpowiednik, ustawiający parametry skryptu \texttt{AudioPitchEstimator}. Funkcją odpowiedzialną za logikę analizy dźwięku jest \texttt{EstimatePitch}, która korzysta z klasy \texttt{AudioPitchEstimator}, na tej podstawia ustawia wartości elementów tekstowych informujących użytkownika o poziomie nastrojenia instrumentu. 
% DONE:zamienione linie kodu
% var nyquistFreq = AudioSettings.outputSampleRate / 2.0f;
% if (!audioSource.isPlaying) return float.NaN;

\begin{lstlisting}[style=sharpcstyle,caption=Funkcja \texttt{Estimate}, label=lst:1]
public float Estimate(AudioSource audioSource)
{
    var nyquistFreq = AudioSettings.outputSampleRate / 2.0f;
    if (!audioSource.isPlaying) return float.NaN;

    audioSource.GetSpectrumData(spectrum, 0, FFTWindow.Hanning);

    for (int i = 0; i < spectrumSize; i++)
    {       
        specRaw[i] = Mathf.Log(spectrum[i] + 1e-9f);
    }
    specCum[0] = 0;
    for (int i = 1; i < spectrumSize; i++)
    {
        specCum[i] = specCum[i - 1] + specRaw[i];
    }

    var halfRange = Mathf.RoundToInt((smoothingWidth / 2) / nyquistFreq * spectrumSize);
    for (int i = 0; i < spectrumSize; i++)
    {    
        var indexUpper = Mathf.Min(i + halfRange, spectrumSize - 1);
        var indexLower = Mathf.Max(i - halfRange + 1, 0);
        var upper = specCum[indexUpper];
        var lower = specCum[indexLower];
        var smoothed = (upper - lower) / (indexUpper - indexLower);        
        specRes[i] = specRaw[i] - smoothed;
    }

    float bestFreq = 0, bestSRH = 0;
    for (int i = 0; i < outputResolution; i++)
    {
        var currentFreq = (float)i / (outputResolution - 1) * (frequencyMax - frequencyMin) + frequencyMin;       
        var currentSRH = GetSpectrumAmplitude(specRes, currentFreq, nyquistFreq);
        for (int h = 2; h <= harmonicsToUse; h++)
        {
           currentSRH += GetSpectrumAmplitude(specRes, currentFreq * h, nyquistFreq);
           currentSRH -= GetSpectrumAmplitude(specRes, currentFreq * (h - 0.5f), nyquistFreq);
        }
        srh[i] = currentSRH;
       
        if (currentSRH > bestSRH)
        {
            bestFreq = currentFreq;
            bestSRH = currentSRH;
        }
    }    
    if (bestSRH < thresholdSRH) return float.NaN;

    return bestFreq;
}
\end{lstlisting}

\begin{itemize}
    \item \texttt{Start} -- funkcja przypisująca przyciskom im działania w wypadku naciśnięcia.
    \item \texttt{ClearColor} -- funkcja pomocnicza, odpowiedzialna za ,,czyszczenie'' elementów graficznych sceny.
\end{itemize}

\subsection{Metronom}

Scena metronomu zawiera 4 przyciski służące do parametryzacji działania metronomu, umożliwiając zwiększenie tempa, zwiększanie lub zmniejszanie liczby taktów. Za logikę odpowiada menadżer sceny z przypiętą klasą \texttt{MetronomeManager}, zawierająca 5 pól obiektów przycisków, pole obiektu tekstowego i kilka zmiennych pomocniczych. Metronom realizowany jest za pomocą wizualizacji poszczególnych taktów za pomocą graficznych elementów w postaci ,,kółek'', które dodawane są do widoku dynamicznie w zależności od ustalonych przez użytkownika wartości. Funkcje jakie implementuje dana klasa to:

\begin{itemize}
    \item \texttt{Start} -- funkcja odpowiadająca za ustawienie zmiennych, obiektów widoku. 
    \item \texttt{AddCircle} -- funkcja wykorzystywana do zwiększenia tempa.
    \item \texttt{DeleteCircle} -- funkcja wykorzystywana do zmniejszenia tempa.
    \item \texttt{RearrangeElements} -- funkcja stworzona do rozmieszczania ,,kółek'' taktów na ekranie.
    \item \texttt{StartMetronome} -- funkcja uaktywniająca metronom.
    \item \texttt{RecolorAll} -- funkcja uruchamiana podczas resetu metronomu, czyszcząca grafiki wizualizacji metronomu.
    \item \texttt{MetronomeCoroutine} -- osobny wątek odpowiedzialny za działanie metronomu, w czasie działania tej funkcji wybijane jest na ekranie tempo.
    \item \texttt{AddBpm} -- funkcja odpowiedzialna za zwiększenie tempa.
    \item \texttt{SubtractBpm} -- funkcja odpowiedzialna za zmniejszenie tempa.
\end{itemize}

Za główną logikę sceny odpowiada funkcja \texttt{MetronomeCoroutine} (listing~\ref{lst:2}). Odgrywa ona dźwięk oznaczający kolejny takt metronomu i zmienia kolory odpowiednich przycisków.

\begin{lstlisting}[style=sharpcstyle,caption=Wątek \texttt{MetronomeCoroutine}, label=lst:2]
IEnumerator MetronomeCoroutine()
{
    float beatDuration = 60f / bpm;
    float noteDuration;
    noteDuration = beatDuration;
    int currentIndex = 0;
    int previousIndex = -1;

    while (true)
    {
      if (circles.Count > 0)
        {
          audioSource.Play(); 
          if (previousIndex != -1 && previousIndex < circles.Count)
          {
              circles[previousIndex].GetComponent<BeatCircle>().changeColorWhite();
          }          
          var currentCircle = circles[currentIndex].GetComponent<BeatCircle>();
          if (currentCircle.imgComponent.color == Color.black)
          {
              currentCircle.changeColorWhite();
          }
          else
          {
              currentCircle.changeColorBlack();
          }
          previousIndex = currentIndex;
          currentIndex = (currentIndex + 1) % circles.Count;
        }
      yield return new WaitForSeconds(noteDuration);
    }
}
\end{lstlisting}

Widok zawiera też przyciski odpowiedzialne za uruchomienie pomocy kontekstowej i opuszczenie bieżącej sceny.

\subsection{Księga akordów}
%DONE : zrobić odwołania poprawne do listingów takie jak w zaleceniach
Scena księgi akordów zawiera element graficzny reprezentujący aktualnie wybrany akord gitarowy, przyciski umożliwiające wybór docelowego akordu, elementy graficzne ułatwiające nawigację po widoku sceny. Funkcje wchodzące w~skład klasy menadżera sceny nazwanego \texttt{ChordChartsManager} to:

\begin{itemize}
    \item \texttt{Start} -- funkcja odpowiedzialna za przygotowanie potrzebnych elementów sceny. Przypisuje grafiki akordów z pomocniczej listy \texttt{Chords} do listy głównej \texttt{ch}. Ustawia wartości zmiennych \texttt{noteIterator}, \texttt{variantIterator} z wartości przechowywanych w skrypcie \texttt{SceneConnector}, łączącej scenę księgi akordów z widokiem wszystkich akordów w postaci diagramu. Przypisuje ona funkcje do odpowiednich przycisków i resetuje wartości obiektów tekstowych.
    \item \texttt{noteIncrement} -- funkcja przypisana do przycisku, odpowiedzialna za zmianę nuty akordu.
    \item \texttt{noteDecrement} -- funkcja przypisana do przycisku, odpowiedzialna za zmianę nuty akordu.
    \item \texttt{variantIncrement} -- funkcja przypisana do przycisku, odpowiedzialna za zmianę wariacji akordu.
    \item \texttt{variantDecrement} -- funkcja przypisana do przycisku, odpowiedzialna za zmianę wariacji akordu.
    \item \texttt{ChangeVariantText} -- funkcja odpowiedzialna za zmianę przypisanej wartości obiektu tekstowego sceny, wykorzystywana w momencie zmiany wariacji akordu.
    \item \texttt{ChangeNoteText} -- funkcja odpowiedzialna za zmianę przypisanej wartości obiektu tekstowego sceny, wykorzystywana w momencie zmiany nuty akordu.
    \item \texttt{ChangeChord} -- funkcja odpowiedzialna za zmianę aktualnie wyświetlanego akordu, zmienia przypisaną grafikę obiektu \texttt{Image}.
    \item \texttt{SavedClicked} -- funkcja pomocnicza służąca zapisowi wybranego akordu.
    \item \texttt{AddToList} -- funkcja dodająca aktualnie wybrany akord do listy przekazywanej następnie do widoku zapisanych akordów.
\end{itemize}

Istotną funkcją jest \texttt{ChangeChord} (\ref{lst:3}), która odpowiada za zmianę wyświetlanej grafiki. Wewnątrz funkcji z racji sposobu zapisu grafik jako pliki .png, nie obsługiwanych domyślnie przez silnik Unity, tworzone są wpierw tekstury, na podstawie których tworzone są grafiki. Te przypisywane są do elementu graficznego znajdującego się na scenie. Widok zawiera też przyciski odpowiedzialne za uruchomienie pomocy kontekstowej i opuszczenie bieżącej sceny.

\begin{lstlisting}[style=sharpcstyle,caption=Funkcja \texttt{ChangeChord}, label=lst:3]
private void ChangeChord()
{
    Texture2D selectedTexture = ch[noteIterator][variantIterator];
    Sprite newSprite = Sprite.Create(selectedTexture, new Rect(0, 0, selectedTexture.width, selectedTexture.height), new Vector2(0.5f, 0.5f));        
    chordImage.sprite = newSprite;        
    image.sprite = newSprite;
}
\end{lstlisting}

\subsection{Trening słuchu}

Scena treningu słuchu zawiera przycisk odpowiedzialny za wybór skali treningowej, po wyborze której pojawiają się przyciski odpowiedzi, wraz z przyciskami odpowiedzialnymi za uruchomienie treningu i jego reset. W skład sceny wchodzą dwie klasy: \texttt{EarTrainerManager} -- przypisany do menadżera sceny realizujący główną logikę sceny oraz \texttt{ButtonIndex} -- klasa pomocnicza przypisana do każdego z przycisków odpowiedzialnych za udzielenie odpowiedzi. Funkcje wchodzące w~skład głównej klasy to:
\begin{itemize}
    \item \texttt{Start} -- funkcja przypisująca obiekt \texttt{audioSource}, odpowiedzialny za odgrywanie dźwięków. Ustawia ona również funkcję \texttt{PlayRandomNote} do przycisku \texttt{playNoteButton}. Dodatkowo przypisuje ona funkcję \texttt{NoteGuessed} do każdego z przycisków odpowiedzialnych za odpowiedź.
    \item \texttt{PlayRandomNote} -- funkcja odpowiedzialna za odegranie losowego dźwięku po naciśnięciu przycisku \texttt{playNoteButton}. Ustawia ona odpowiedni index dla odegranej nuty, wykorzystywany następnie podczas zweryfikowania odpowiedzi podanej przez użytkownika.
    \item \texttt{NoteGuessed} -- funkcja weryfikująca odpowiedź użytkownika. Ustawia ona obiekty tekstowe sceny, w zależności od tego czy odpowiedź była poprawna, czy też nie. 
    \item \texttt{DeactivateObjects} -- funkcja odpowiedzialna za wyłączenie obiektów graficznych sceny.
    \item \texttt{ResetTraining} -- funkcja dezaktywująca aktualny trening, przenosząc użytkownika do widoku początkowego.
    \item \texttt{SelectedScaleTreining*} -- funkcje odpowiadające za ustawienie parametrów sceny w zależności od wybranej skali treningu.
\end{itemize}

Za odegranie nuty, a następnie sprawdzenia poprawności odpowiedzi odpowiedzialny jest główna klasa (listing~\ref{lst:4}).

\begin{lstlisting}[style=sharpcstyle,caption=Funkcje \texttt{PlayRandomNote} i \texttt{NoteGuessed}, label=lst:4]
void PlayRandomNote()
{
    feedbackText.text = "Odgadnij"; 
    correctNoteIndex = Random.Range(0, noteClips.Count);
    
    audioSource.clip = noteClips[correctNoteIndex];
    audioSource.Play();
}

void NoteGuessed(Button button)
{
   if (button.GetComponent<ButtonIndex>().buttonNumber == correctNoteIndex)
   {
       feedbackText.text = "poprawnie";
   }
   else
   {
       feedbackText.text = "niepoprawna odpowiedź";
   }
}
\end{lstlisting}

Klasa pomocnicza przechowuje jedynie jedną zmienną, wykorzystywaną w momencie weryfikacji odpowiedzi udzielonej przez użytkownika, jest nią: \texttt{buttonNumber}. Widok zawiera też przyciski odpowiedzialne za uruchomienie pomocy kontekstowej i opuszczenie bieżącej sceny.

\subsection{Koło kwintowe}

Scena koła kwintowego zawiera 24 przyciski, każdemu z nich przypisana została odpowiednia nuta. W kodzie menadżera sceny utworzone zostały dwie listy, przechowujące przyciski koła zewnętrznego oraz koła wewnętrznego. Kod zawiera w sobie 6 funkcji:

\begin{itemize}
\item \texttt{Start} -- funkcja używana podczas tworzenia obiektu, służy ona przygotowaniu wszystkich przycisków oraz przypisania im działania przy ich naciśnięciu, w tym przypadku przypisuje ich działanie do funkcji \texttt{DisplayScale}
\item \texttt{DisplayScale} -- funkcja odpowiedzialna jest za wyświetlenie skali majorowej, oraz 3 akordów minorowych pasujących do wybranej przez użytkownika nuty. 
\item \texttt{GetColorTint} -- funkcja pomocnicza generująca kolejne odcienie kolorów, dla funkcji \texttt{DisplayScale}.
\item \texttt{GetMajorScaleIndices} -- funkcja pomocnicza generująca interwały kolejnych nut w skali majorowej.
\item \texttt{ChangeButtonColor} -- funkcja pomocnicza odpowiedzialna za zmianę koloru odpowiedniego przycisku.
\item \texttt{ClearColors} -- funkcja pomocnicza czyszcząca kolory wszystkich przycisków.
\end{itemize}

Najistotniejsza funkcja, czyli \texttt{DisplayScale} zademonstrowano poniżej (\ref{lst:5}), pominięta część funkcji to fragment kodu odpowiedzialny jedynie za zmianę koloru odpowiednich elementów.

\begin{lstlisting}[style=sharpcstyle,caption=Funkcja \texttt{DisplayScale}, label=lst:5]
public void DisplayScale(int noteIndex)
{
    ClearColors();
    List<int> scaleIndices = GetMajorScaleIndices(noteIndex);
    for (int i = 0; i < scaleIndices.Count; i++)
    {
        int index = scaleIndices[i];
        if (index < outerButtons.Count)
        {
            Color tint = GetColorTint(i);
            ChangeButtonColor(outerButtons[index], tint);
        }
    }
    ...         
}
\end{lstlisting}

Widok zawiera też przyciski odpowiedzialne za uruchomienie pomocy kontekstowej i opuszczenie bieżącej sceny.

\section{Diagram akordów}

Diagram akordów jest widokiem, do którego można dostać się jedynie ze sceny księgi akordów. Przedstawia on wszystkie akordy zawarte w księdze akordów, aczkolwiek w widoku ogólnym. Funkcje jakie zawiera klasa głównego menadżera sceny odpowiedzialne są za uaktywnianie i~dezaktywowanie poszczególnych widoków akordów.

\begin{itemize}
    \item \texttt{DiagramNextPage} -- funkcja służąca do przewijania aktualnie wyświetlanych akordów
    \item \texttt{DiagramPreviousPage} -- funkcja służąca do przewijania aktualnie wyświetlanych akordów
    \item \texttt{UpdateDiagramDisplay} -- funkcja służąca do aktualizacji aktualnie wyświetlanych akordów
    \item \texttt{DiagramSelected} -- Funkcja odpowiedzialna za zapis wybranego akordu i przekazanie informacji do księgi akordów.
\end{itemize}

\section{Zapisane akordy}

Widok zapisanych akordów zawiera w sobie 3 elementy graficzne reprezentujące wybrane w księdze akordów akordy. Kod kontrolera sceny zawiera tylko jedną funkcję \texttt{Start}, odpowiedzialną za wyświetlenie wcześniej zapisanych akordów. Widok zawiera też przycisk odpowiedzialny za opuszczenie bieżącej sceny.