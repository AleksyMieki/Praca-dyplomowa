\pdfbookmark[0]{Streszczenie}{streszczenie.1}

\begin{abstract}
Celem pracy było utworzenie aplikacji desktopowej służącej do rozwijania potencjału muzycznego gitarzystów. Początkowo przedstawiono wstępny zamysł projektu, zapoznano czytelnika z potrzebnymi terminami i przedstawiono istotę projektu. Aby dobrze podejść do realizacji projektu, analizie poddano 2 istniejące na rynku aplikacje oferujące narzędzia dla gitarzystów. Na podstawie analizy aplikacji sporządzono listę wymagań funkcjonalnych, jak i niefunkcjonalnych. Ustalono, że gotowy projekt zawierać ma: metronom, koło kwintowe, księgę akordów z diagramem i możliwością zapisu akordów, trening słuchu oraz stroik. Dalsza część dokumentu poświęcona została planowi gotowej aplikacji. Zdecydowano się na układ widoków oraz zawartość poszczególnych scen. Na podstawie planu projektu przystąpiono do implementacji aplikacji za pomocą silnika Unity. W dokumencie zawarto szczegóły implementacji poszczególnych scen. Etapem końcowym było przeprowadzenie testów aplikacji w oparciu o wbudowane w silnik Unity narzędzie do przeprowadzania testów jednostkowych. Dodatkowo przeprowadzono testy akceptacyjne aplikacji. Projekt zakończył się na zrealizowaniu założonych celów.
\end{abstract}
\mykeywords

{
\selectlanguage{english}
\begin{abstract}
The aim of the work was to create a desktop application designed to develop the musical potential of guitarists. Initially, the preliminary project concept was presented, the reader was introduced to the necessary terminology, and the essence of the project was explained. To approach the project implementation properly, two existing applications offering tools for guitarists were analyzed. Based on the application analysis, a list of functional and non-functional requirements was drawn up. It was established that the finished project would include: a metronome, circle of fifths, a chord book with a diagram and the ability to save chords, ear training, and a tuner. The further part of the document was devoted to the plan of the finished application, and the layout of views and the content of individual scenes were decided upon. Based on the project plan, implementation of the application using the Unity engine was commenced, and the document included details of implementing individual scenes. The final stage was conducting application tests based on the unit testing tool built into the Unity engine. Additionally, acceptance tests of the application were carried out. The project concluded with the achievement of the set goals.
\end{abstract}
\mykeywords
}
