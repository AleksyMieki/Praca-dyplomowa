\chapter{Podsumowanie}

Praca dyplomowa miała na celu utworzenie aplikacji desktopowej dla gitarzystów, służącej do nauki i doskonalenia umiejętności muzycznych. Zakres przeprowadzonych działań obejmował kompleksowy przegląd dostępnych rozwiązań, sporządzenie szczegółowych wymagań aplikacji, opracowanie projektu, implementację oraz serię testów.

Podczas wstępnej analizy przebadano dwie istniejące aplikacje, oceniając ich użyteczność i~dostępne narzędzia. Na podstawie przeprowadzonej analizy sporządzono listę kluczowych funkcji dla projektowanej aplikacji, którymi były: koło kwintowe, księga akordów, stroik gitarowy, metronom oraz trening słuchu.

Projekt został zrealizowany przy użyciu silnika Unity z wykorzystaniem języka programowania C\#. Wszystkie zaplanowane funkcje zostały poprawnie zaimplementowane i potwierdzone poprzez manualne testy aplikacji. Narzędzia umożliwiają kompleksowy trening muzyczny: metronom wspomaga ćwiczenie rytmu, księga akordów i koło kwintowe rozwijają teorię muzyczną, a trening słuchu doskonali umiejętności rozpoznawania dźwięków. Kluczowym elementem jest stroik, pozwalający na dokładne nastrojenie instrumentu.

Podczas implementacji napotkano wyzwania techniczne, zwłaszcza przy tworzeniu funkcji stroika. Wykorzystywana biblioteka analizy dźwięku wymagała modyfikacji parametrów początkowych w celu dokładnego wykrywania dźwięków. Głównym problemem było nieprawidłowe rozpoznawanie częstotliwości podstawowej, gdzie dla dźwięku A algorytm zwracał jego harmoniczne. Rozwiązaniem było ustawienie górnej granicy tolerancji, co znacząco poprawiło dokładność detekcji. Kolejnym napotkanym problemem przy realizacji funkcji stroika było zbyt duże obciążenie podczas ciągłej analizy dźwięku. Rozwiązano ten problem poprzez nagrywanie dźwięku, odtwarzanie go, gdy scena jest otwarta, a następnie wywoływanie funkcji analizy dźwięku w określonych większych odstępach czasowych. Do mniejszych problemów można zaliczyć nieprawidłowe działanie metronomu, spowodowane nieodpowiednim wywoływaniem wątku odpowiadającego za wywoływanie taktu. Ten problem udało się rozwiązać, tworząc zmienną przechowującą odwołanie do uruchomionego wątku, który był w razie potrzeby zatrzymywany i uruchamiany na nowo.

Perspektywy rozwoju aplikacji obejmują kilka potencjalnych ulepszeń takich jak: rozszerzenie funkcjonalności stroika o możliwość wyboru różnych instrumentów, wprowadzenie większej ilości skal do narzędzia treningu słuchu, implementacja dodatkowych narzędzi edukacyjnych, np. treningu pamięci muzycznej poprzez wizualne prezentowanie nut do rozpoznania, wprowadzenie zaawansowanych funkcji kompozycyjnych z wykorzystaniem algorytmów sztucznej inteligencji do automatycznego tworzenia podkładów muzycznych.

Podsumowując, projekt stanowi kompleksowe narzędzie wspierające naukę gry na gitarze, łącząc elementy praktyczne i teoretyczne, z wyraźnym potencjałem dalszego rozwoju.
